\documentclass[29pt,a4paper]{moderncv}

% moderncv themes
%\moderncvtheme[blue]{casual}                 % optional argument are 'blue' (default), 'orange', 'red', 'green', 'grey' and 'roman' (for roman fonts, instead of sans serif fonts)
\moderncvtheme[green]{banking}                % idem

\usepackage[T1]{fontenc}
% character encoding
\usepackage[utf8x]{inputenc}               	% replace by the encoding you are using
\usepackage[italian]{babel}
\usepackage{color}

% adjust the page margins
\usepackage[scale=0.8]{geometry}
\recomputelengths                          	% required when changes are made to page layout lengths

\fancyfoot{} % clear all footer fields
\fancyfoot[L,RO]{\thepage}           		% page number in "outer" position of footer line
\fancyfoot[R,LO]{\footnotesize} 			% other info in 


\begin{document}
\section{\textbf{Table of Contents:}}
\begin{tabbing}
\\\textbf{Subject}: ~~~~~~~~~~~~~~~~~~~~~~~~~~~~~~~~~~~~~~~~~~~~~~~~~~~~~~~~~~~~~~~~~~~~~~~~~~~~~~~~~~~~~~~~~~~~~~~~~~~~~~~~~~\= \textbf{Page}:
\\\newline
	\\1. Business/Client Perspective \> ~~~~~~\= 2
		\\~~~~~~~~\= 1.1 Background \>  2
		\\~~~~~~~~\= 1.2 Business Opportunity \> 2
		\\~~~~~~~~\= 1.3 Customer and Market Requirements \> 2
	\\2. Vision \> \> 3
	\\3. Scope and Limitations \> \> 3
		\\~~~~~~~~\= 3.1 Initial Minimal Release \> 3
		\\~~~~~~~~\= 3.1 Subsequent Releases \> 3 
	\\4. High Level Non-Functional Requirement \> \> 3
		\\~~~~~~~~\= 4.1 Core Quality Requirements \> 3
		\\~~~~~~~~\= 4.2 Infrastructure  \> 4
	\\5. Project Sucess Factors \> \> 4
		\\~~~~~~~~\= 5.1 Driver \> 4
		\\~~~~~~~~\= 5.2 Constraints \> 4
		\\~~~~~~~~\= 5.3 Important\> 4
		\\~~~~~~~~\= 5.4 Nice-To-Have\> 4
		\\~~~~~~~~\= 5.4 Optional Extras\> 4

\end{tabbing}
\newpage
	%\maketitle
	%\vspace{-10mm}
	%Section
	\section*{\textbf{1. Business and Client Perspective}}
	\vspace{4mm}

	\vspace{5mm}
	
		\textbf{1.1 Background}
		\vspace{4mm}
			\\The need for this project comes from the fact that sending mathematical equations in instant messages are rather difficult and a tedious task. There are no instant messaging services that support the rendering of mathematical equations into the correct format and mathematical representation.
			
			\parindent 5mm There is no current system that will be replaced. Instead different systems and services will be used within the new system that we will design and develop for this project. The systems that we will use will be the XMPP client and MimeTeX for rendering. The TeXchat system will use the functionality of yaxim as basis and reference.
		\vspace{5mm}
		
		\noindent \textbf{1.2 Business Opportunity}
		\vspace{4mm}
			\\The TeXchat application will be open source. The closest competitor for the TeXchat proposal would be Pidgintex, although Pidgintex is desktop based and TeXchat will be mobile oriented. 
			
			\parindent 5mm The reason for creating such an application would be to make it easier for mathematicians and people interested in the mathematical field to communicate and collaborate, sending equations and discussing the equations.
			
			\parindent 5mm The specific target market would be mathematicians, lecturers and also students who are studying mathematics or have a certain interest in mathematics.
		\vspace{5mm}
		%Section
		
		\noindent \textbf{1.3 Customer and Market Requirements}
		\vspace{4mm}
			\\A customer for this software product will typically be any person who wishes to communicate in an informal manner about information and data of a mathematical nature.  Thus for this application to be a success our customer base needs to include people with the relevant background and knowledge regarding information of a mathematical nature.  These people typically include students in one of the various fields of engineering, information technology and science. 
			
			\parindent 5mm It will be highly useful to such students as well as any professional who wishes to share ideas, problems and research in any field with a mathematical component, and thus our target audience is very broad.
			
			\parindent 5mm Since our audience will consist of a very broad group, one of our main customer requirement concerns is the level of basic mathematical knowledge across the various fields. Therefore we will assume a very basic understanding of mathematical knowledge and fundamental concepts of calculus, as this will be adequate according to our client Mr. van Heerden.
			
			\parindent 5mm We will require 2 to 4 individuals, either fellow students or lecturers, to introduce our application to our target market. The aim is to build prototypes, or product demonstrations in product  build - release cycles to assess how our product will be placed in our target market, and how successful it will be in meeting their needs.
			
			\parindent 5mm Furthermore this application requires a professional and easy to use interface for our target market since it will consist of people in a higher educational environment.
		\vspace{5mm}
	
\newpage
	%Section
	\noindent\section*{\textbf{2. Vision}}
	\vspace{4mm}
		\noindent Our vision is to create a successful and fully functional chat application, that supports users with their communications in a mathematical environment.
\newline
		\\Every design and implementation decision will be made with the following vision and mindset:
		\begin{itemize}
			\item Will a person studying mathematics be able to use this?
			\item What will the essential requirements be for them to use the application to fulfill their needs?\\
		\end{itemize}\\
		
\\\newline
		\noindent All the decisions will basically revolve around the fact that this chat application will be used by students, lecturers and other professionals in a mathematical environment to support them with their communications in this field.
	\vspace{5mm}
	
	
	%Section
	\section*{\textbf{3.Scope and Limitations}}
	\vspace{4mm}

		\noindent \textbf{3.1 Initial Minimal Release}
		\vspace{4mm}
			\\Our initial release will support communication between two users.  They will be enabled to log into the system using a encrypted password and username and they will each have a profile that they can edit and improve as they wish.  
			
			These users will be able to communicate using normal text, and a variety of mathematical equations, which they will be able to view and edit before they are sent.
		\vspace{5mm}
		
		\noindent \textbf{3.2 Subsequent Releases}
		\vspace{4mm}
			\\In subsequent releases, we will extend our application to support two or more users, allowing multiple conversations to take place simultaneously.
		\vspace{5mm}

	\vspace{5mm}	
	
	\section*{\textbf{4. High-Level Non-Functional Requirements}}
	\vspace{4mm}
	
		\noindent \textbf{4.1 Core Quality Requirements}\\
		\begin{itemize}
			\item Number of concurrent users:
			\begin{itemize}
				\item Initially the number of concurrent users would be 2, this will be the 2 users having the conversation. This will be scaled up as the system evolves throughout the development cycle.
			\end{itemize}
			
			\item Security:
			\begin{itemize}
				\item Users’ passwords must be encrypted.
				\item Chats can be confidential using XMPP/SSL (nice-to-have).
			\end{itemize}
			
			\item Auditability:
			\begin{itemize}
				\item All of the chats must be logged.
				\item Logs older than a certain period will be deleted automatically, enhancing performance.
			\end{itemize}
			
			\item Installability:
			\begin{itemize}
				\item The user must be able to install the application via the Android PlayStore, or installing it manually.
			\end{itemize}
		\end{itemize}
		\vspace{4mm}

		\vspace{5mm}

\newpage
		\noindent \textbf{4.2 Infrastructure}
		\vspace{4mm}
			\begin{itemize}
				\item Hardware:
				\begin{itemize}
					\item The hardware on which the application will be tested on will basically be any mobile device that runs an Android operating system
					\item Mobile device requirements: CPU: 528 MHz, RAM: 384MB.
					\item The above requirements are just an estimate for the most basic device that will be able to run the application.
				\end{itemize}
				
				\item Software:
				\begin{itemize}
					\item The mobile device will have to run Android v2.2 or higher.
					\item Software we will be using to develop for the device will be: Eclipse IDE and the Android SDK.
				\end{itemize}
								
				\item Integration with other systems:
				\begin{itemize}
					\item Yaxim will be used as a reference.
					\item Mimetex will be used to parse the LaTeX code.
					\item XMPP/Jabber client will be the server through which the clients will communicate.
				\end{itemize}
			\end{itemize}
		\vspace{5mm}

		
	\section*{\textbf{5. Project Success Factors}}
	\vspace{4mm}
	
		\noindent \textbf{5.1 Driver}
		\vspace{4mm}
			\begin{itemize}
				\item The driver for this project would be successfully parsing the mathematical equations using LaTeX and then sending the instant message to the other client on the receiving end.
			\end{itemize}
			\vspace{5mm}
		
		\noindent \textbf{5.2 Constraints}
			\vspace{4mm}
			\begin{itemize}
				\item Showing the preview of the entered equation before it is sent to the client on the receiving end.
				\item The ability for the application to export a conversation as a LaTeX file.
				\item The application must be installable on a mobile device.
			\end{itemize}
			\vspace{5mm}

		\noindent \textbf{5.3 Important}
			\vspace{4mm}
			\begin{itemize}
				\item Encryption of passwords, this does not directly benefit the user, but is extremely important security wise. 
			\end{itemize}
			\vspace{5mm}
		
		\noindent \textbf{5.4 Nice-To-Have}
		\vspace{4mm}
			\begin{itemize}
				\item Chats that will be encrypted, making the conversation confidential.
				\item A user’s profile may be customizable to their preference.
			\end{itemize}	
		\vspace{5mm}
				
		\noindent \textbf{5.5 Optional Extra}
		\vspace{4mm}
			\begin{itemize}
			\item Designing the infrastructure of the system in such a manner that it will be easy to port to a desktop application (this will only be implemented after the entire project has been completed successfully and the schedule allows for this optional extra).
			\end{itemize}	
		\vspace{5mm}
\end{document}